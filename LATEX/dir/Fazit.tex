\chapter{Fazit}
\label{chapter:Fazit}
\index{Fazit der Arbeit|(}

Beim Kickoff Meeting und der Bekanntgabe der Eckdaten f�r die Programmieraufgabe kamen einigen der Teilnehmern, mich eingeschlossen, die Aufgabenstellung als nicht l�sbar vor. Als nicht Programmierer und ohne Kenntnisse in der C-Programmiersprache schien es als absolut nicht machbar.\\
Nichts desto trotz gab es keine andere M�glickkeit, mindestens einen Teil der Mission Impossible zu l�sen.

Da ich wusste, dass diese Arbeit ein harter Brocken werden wird, habe ich mich fr�h hingesetzt und mit dem Grundkonzept angefangen. W�hrend des Semesters und dem Modul \flqq Systemsoftware\frqq\ bekam ich jedoch einen Einblick in die C-Programmiersprache und der Server nahm von Woche zu Woche mehr an Gestalt an.\\
Da ich nicht wusste, ob ich die Aufgabe meistern konnte, habe ich mich entschieden, meine aktuellen St�nde immer zu Dokumentieren (siehe Kapitel \ref{chapter:Aufw�nde}) und auf github zu aktualisieren. So w�re immerhin ein roter Faden in der Arbeit zu sehen.\\
Zu meiner eigenen Verwunderung wuchs der Server- und der Clientteil der Aufgabe zu einem Konstrukt, das mir selber Freude bereitete, da ich sehen konnte, wie Fortschritte erzielt wurden.

In den letzten Wochen vor der Abgabe wurde die Dokumentation detaillierter und der Server auf seine Funktionalit�ten besser gepr�ft. So konnten noch einige \flqq segmentation faults\frqq\ behoben werden. Diese traten vor allem auf, wenn ein Befehl wie CREATE ohne zust�zliche Argumente an den Server gestellt wurden.

Durch den eigenen Ehrgeiz, die w�hrend dem Kickoff gestellten Anforderungen zu erf�llen, enstand schlussendlich die finale Version des Server. Sie hat zwar keinen Anspruch auf Fehlerfreiheit, funktioniert jedoch sehr gut. Nicht zu vergessen sind jedoch die zus�tzlich aufgewendenten Stunden, um der C-Programmiersprache m�chtig zu werden.\\
Die Einarbeitungszeit, die Programmierung und die Dokumentation haben die doppelte Zeit in Anspruch genommen als f�r dieses Seminar veranschlagt war.
  
Trotz des viel gr�sseren Zeitaufwandes hat die Arbeit grunds�tzlich Spass gemacht. Durch diese Arbeit konnte sehr viel Wissen angeeignet werden. Einzig das Debugging war teils sehr m�hsam, falls etwas nicht funktionierte.\\
M�sste diese Arbeit mit dem angeeigneten Wissen (durch Unterricht Systemsoftware und eigenes Erarbeiten f�r diese Semiararbeit) nochmals wiederholt werden, denke ich, dass die Aufgabe im Zeitrahmen der vorgeschlagenen 50-60 Stunden gemeistert werden kann.



\index{Fazit der Arbeit|)}