\chapter{Einleitung}
\label{sec:Einleitung}
%============== N E W  ==== S E C T I O N  ======== 
\section{Quellenangaben}
\label{subsec:Quellenangaben}

Der Quellcode der gesammten Arbeit wurde vom Verfasser selber geschrieben. Nat�rlich k�nnen sich einzelne Fragmente im Internet gefunden werden, da dies in der Programmierung immer der Fall sein kann. Es wurde jedoch kein Quellcode vom Internet �bernommen, sondern Ideen gesammelt und standardisierte Funktionen benutzt. Hier soll als Beispiel ein memcpy erw�hnt werden.\\
Der einzige Ort von dem Code kopiert wurde ist das github Repository des Dozenten Karl Brodowsky, welcher das Modul \flqq Systemsoftware\frqq\ durchf�hrte (siehe Quelle \cite{githubdozent}). Diese kopierten Dateien sind jedoch im Header als solche erkennbar.

F�r die Umsetzung der Arbeit wurden sehr viele kleinere Probleme wie z.B.: wie kopieren ich einen String? im Internet gesucht. Es ist schlicht nicht m�glich, diese Quellen alle anzugeben.\\
Die verwendeten B�cher sind im Quellenverzeichnis am Schluss dieser Arbeit zu finden.



%============== N E W  ==== S E C T I O N  ======== 
\section{Rahmenbedingungen} \index{Rahmenbedingungen|(}
\label{subsec:Ausganslage}
Die Aufgabenstellung und die Rahmenbedingungen wurden �ber Github (\url{https://github.com/telmich/zhaw_seminar_concurrent_c_programming}) ver�ffentlicht.\\
Anbei ein Auszug aus den wichtigsten Eckdaten und Anforderungen:
\index{Rahmenbedingungen|)}
%============== N E W  ==== S U B - S E C T I O N  ======== 
\subsection{Geplante Termine}
\begin{table}[h!]
\begin{minipage}{10cm}
\begin{tabular}[t]{|l|l|} \hline
 \cellcolor{darkgrey} &  \cellcolor{darkgrey}  \\ 
\cellcolor{darkgrey} \multirow{-2}{3cm}{\textbf{Datum}} &
\cellcolor{darkgrey} \multirow{-2}{6cm}{\textbf{Beschreibung}}  \\  \cline{1-2}
13.03.2014 & Kick-Off Meeting   \\ \cline{1-2}
24.06.2016 & Abgabe der schriftlichen Arbeit (1 Woche vor Pr�sentation) \\ \cline{1-2}
01.07.2014 & Pr�sentation der Arbeit\\   \cline{1-2}
02.07.2014 & optionale Teilnahme an anderen Pr�sentationen\\   \cline{1-2}
03.07.2014 & optionale Teilnahme an anderen Pr�sentationen\\   \cline{1-2}
21.07.2014 & Notenabgabe\\   \cline{1-2}
\end{tabular}
\caption{geplante Termine}
\label{tab:geplante_Termine}
\end{minipage}
\end{table}
%============== N E W  ==== S U B - S E C T I O N  ======== 
\subsection{Administratives}
\begin{itemize}
\item Abgabe Arbeit via git repository auf github.com
\item Zum Zeitpunkt \frqq Abgabe Arbeit\frqq\ werden alle git repositories geklont, �nderungen danach werden *NICHT* f�r die Benotung beachtet.
\end{itemize}
%============== N E W  ==== S U B - S E C T I O N  ======== 
\subsection{Abgabebedingungen}
\begin{itemize}

\item git repo auf github vorhanden
\item Applikation lauff�hig unter Linux
\item Nach \grqq make\grqq\ Eingabe existiert
\begin{itemize}
\item \grqq run\grqq: Binary des Servers
\begin{itemize}
\item[-] Sollte nicht abst�rzen / SEGV auftreten
\end{itemize}
\item \grqq test\grqq: Executable zum Testen des Servers
\end{itemize}
\item \grqq doc.pdf\grqq: Dokumentation
\item Einleitung
\item Anleitung zur Nutzung
\item Weg, Probleme, L�sungen
\item Fazit
\item Keine Prosa - sondern guter technischer Bericht
\item Deutsch oder English m�glich
\end{itemize}
%============== N E W  ==== S U B - S E C T I O N  ======== 
\subsection{Vortrag / Pr�sentation}
\begin{itemize}
\item 10 - 15 Minuten + 5 Minuten Fragen
\item Richtzeiten:
\begin{itemize}
\item Einleitung (2-3) min
\item Weg, Probleme, L�sungen (4-10) min
\item Implementation zeigen (2-5) min
\item Fragen (2-5) min
\end{itemize}
\item Vortrag ist nicht (nur) f�r den Dozenten
\end{itemize}
%============== N E W  ==== S U B - S E C T I O N  ======== 
\subsection{Lernziele}
\begin{itemize}
\item Die Besucher des Seminars verstehen was Concurrency bedeutet und welche Probleme und L�sungssans�tze es gibt.
\item Sie sind in der Lage Programme in der Programmiersprache C zu schreiben, die auf gemeinsame Ressourcen gleichzeitig zugreifen.
\item Das Seminar setzt Kenntnisse der Programmiersprache C voraus.
\end{itemize}
%============== N E W  ==== S U B - S E C T I O N  ======== 
\subsection{Lerninhalte}
\begin{itemize}
\item Selbstst�ndige Definition des Funktionsumfangs des Programmes unter Ber�cksichtigung der verf�gbaren Ressourcen im Seminar.
\item Konzeption und Entwicklung eines Programms, das gleichzeitig auf einen Speicherbereich zugreift.
\item Die Implementation erfolgt mithilfe von Threads oder Forks und Shared Memory (SHM).
\end{itemize}

%============== N E W  ==== S E C T I O N  ======== 
\section{Das Projekt}
\label{sec:Das Projekt}
\begin{itemize}
\item kein globaler Lock (!)
\item Kommunikation via TCP/IP (empfohlen) - Wahlweise auch Unix Domain Socket
\item fork + shm (empfohlen)
\begin{itemize}
\item oder pthreads
\item f�r jede Verbindung einen prozess/thread
\item Hauptthread/prozess kann bind/listen/accept machen
\end{itemize}
\item Fokus liegt auf dem Serverteil
\begin{itemize}
\item Client ist haupts�chlich zum Testen da
\item Server wird durch Skript vom Dozent getestet
\end{itemize}
\item Wenn die Eingabe valid ist, bekommt der Client ein OK
\begin{itemize}
\item Locking, gleichzeitiger Zugriff im Server l�sen
\item Client muss *nie* retry machen
\end{itemize}
\item Protokolldefinitionen in protokoll/
\item Alle Indeces beginnen bei 0
\item Debug-Ausgaben von Client/Server auf stderr
\end{itemize}

\textbf{Fileserver}
\begin{itemize}
\item Dateien sind nur im Speicher vorhanden
\item Das echte Dateisystem darf NICHT benutzt werden
\item Mehrere gleichzeitige Clients
\item Lock auf Dateiebene
\end{itemize}


%============== N E W  ==== S E C T I O N  ======== 
\section{Ausgangslage}
\label{sec:Ausgangslage}
Die Aufgabenstellung, wie sie oben beschrieben ist, ist f�r einen nicht Programmierer gem�ss Dozent eine grosse Herausforderung. Mindestens vier Studenten, zu denen auch ich z�hle, haben ihre Bedenken ge�ussert, dass die Anforderungen der Aufgabenstellung fast nicht zu erreichen w�ren. Ein Informatiker, dessen Zuhause ist das Programmieren ist geschweige denn die Sprache \flqq C\frqq, wird f�r eine minimalistische L�sung bei weitem mehr Stunden ben�tigen als die 60 Stunden, welche f�r dieses Seminararbeit gedacht sind.\\
Damit f�r den Dozenten besser ersichtlich ist wie viel Zeit f�r welche Teile der Arbeit aufgewendet wurden, sind im Kapitel \refTC{chapter:Aufw�nde} die Zeiten erfasst und ausgewiesen.



