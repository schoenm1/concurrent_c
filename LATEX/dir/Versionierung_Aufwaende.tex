
%==============  N E W  ==== C H A P T E R ==============%
\chapter{Versionierung}
\label{chapter:Versionierung}

\begin{table}[h!]
\begin{minipage}{10cm}
\begin{tabular}[t]{|l|l|l|} \hline
\cellcolor{darkgrey} &  \cellcolor{darkgrey} &  \cellcolor{darkgrey}  \\ 
\cellcolor{darkgrey} \multirow{-2}{1.5cm}{\textbf{Version}} &
\cellcolor{darkgrey} \multirow{-2}{1.5cm}{\textbf{Datum}}  &
\cellcolor{darkgrey} \multirow{-2}{8cm}{\textbf{Beschreibung}} \\  \cline{1-3}
V0.1 & 15.03.2014 & Ersterstellung Dokument \\   \cline{1-3}
V0.2 & 17.03.2014 & Einleitung, Ausgangslage \\   \cline{1-3}
V0.3 & 07.04.2014 & Grundger�st, Konzept \\   \cline{1-3}
V0.4 & 08.04..2014 & Implementierung Argument-�berpr�fung (LogLevel) \\   \cline{1-3}
V0.5 & 13.04.2014 &Erweitern Server (Argument-�berpr�fung) \\   \cline{1-3}
V0.6 & 01.05.2014 & Speicherverwaltung mit Buddy \\   \cline{1-3}
V0.7 & 02.05.2014 & Client Connection to Server \\   \cline{1-3}
V0.8 & 02.05.2014 & TCP Connection Protocol, Loglevel \\   \cline{1-3}
V0.9 & 02.05.2014 & CREATE, LIST, LOG, Fehlebehebungen \\   \cline{1-3}
V1.0 & 02.05.2014 & One Thread per Client, Refactoring, Commenting Code \\   \cline{1-3}
V1.1 & 02.05.2014 & Client Connection to Server \\   \cline{1-3}
V1.2 & 02.05.2014 &Implementing RWLock for Creating File \\   \cline{1-3}
V1.3 &03.05.2014 & Loglevel und eigenes TCP-Protokoll\\ \cline{1-3}
V1.3b & 04.05.2014 &  CREATE und Log Verbesserungen\\  \cline{1-3}
V1.3c & 05.05.2014 &   LIST shm / Fehlersuche CREATE \\ \cline{1-3}
V1.4 & 06.05.2014 & dynamisches TRACE\_LOG\\ \cline{1-3}
V1.5 & 26.05.2014 & \\   \cline{1-3}
V1.5b & 27.05.2014 & \\   \cline{1-3}
V1.5c & 28.05.2014 & \\   \cline{1-3}
V1.6 & 31.05.2014 & \\   \cline{1-3}
V1.6b & 01.06.2014 & \\   \cline{1-3}
V1.7 & 02.06.2014 & \\   \cline{1-3}
V1.8 & 10.06.2014 & \\   \cline{1-3}



\end{tabular}
\end{minipage}
\caption{Versionierung Dokumentation}
\label{tab:Versionierung}
\end{table}







%==============  N E W  ==== C H A P T E R ==============%
\chapter{Aufw�nde}
\label{chapter:Aufw�nde}


\begin{longtable}{| l | c | l |} 
\hline
\cellcolor{darkgrey} &  \cellcolor{darkgrey} &  \cellcolor{darkgrey}  \\ 
\cellcolor{darkgrey} \multirow{-2}{1.5cm}{\textbf{Datum}} & 
\cellcolor{darkgrey} \multirow{-2}{1.5cm}{\textbf{Zeit}}  &
\cellcolor{darkgrey} \multirow{-2}{8.5cm}{\textbf{Beschreibung}} \\  \cline{1-3}

 \multirow{4}{1cm}{15.03.2014} &  \multirow{4}{1cm}{3.75h} & -Ersterstellung Dokumentation\\   
 & & -Github Repo erstellen \\  
 & & -Einlesen Buch Kapitel 15 (Semaphore, Shared Memory, \dots)\\
 & & -Erstellen Debian VM \\ \cline{1-3}

 \multirow{2}{1cm}{17.03.2014} &  \multirow{2}{1cm}{0.5h} & -Dokumentation: Einleitung\\
  & & (Rahmenbedingungen, Projekt, Ausgangslage)\\   \cline{1-3}

 \multirow{1}{1cm}{07.04.2014} &  \multirow{1}{1cm}{1.75h} & -Grundger�st erstellen, LOG-LEVEL definieren \\ \cline{1-3}

 \multirow{2}{1cm}{08.04.2014} &  \multirow{2}{1cm}{3h} & -Parsing Argumente bei Programmstart\\
 & & -Log-Level Implementierung\\ \cline{1-3}

 \multirow{8}{1cm}{13.04.2014} &  \multirow{8}{1cm}{1.75h} & -Auslagern Funktionen in externe .h Dateien\\
& & -Anpassen Argument-Validierung: wenn Argument mehr als\\
& & \ 1mal vorkommt, wird es ignoriert\\
& & -bei nicht setzen des LogLevel wird default LogLevel initialisiert\\
&  & -Erstinitialisierung TCP-Server: wartet auf Verbindung\\
& & \  von Client\\
& & -Probleme: \#define von LOG LEVELS in log-Level.h\\
& & \ sind nicht sichtbar in \grqq server.h\grqq.\\ \cline{1-3}

15.04.2014 &1h& -Installieren von e-UML -> funktioniert nur mit Java ;-(\\  \cline{1-3}

 \multirow{17}{1cm}{01.05.2014} &  \multirow{17}{1cm}{9h} & -Degub mit \#define funktioniert nicht.\\
& & -> Einlesen in andere M�glichkeiten\\
& & -gem�ss R�cksprache mit anderen Studenten sollte nicht ein File\\
& & \ wirklich eingelesen werden (von HDD ge�ffnet und Stream \\
& &  \ �bermittelt), sondern lediglich mit dem Filenamen und\\
& & \ Gr�sse angelegt werden im Shared Memory\\
& & -	Versuch, Control Shared Memory zu l�sen mit einem\\
& & \ \ Buddy System\\
& & \textbf{Fazit Arbeiten}:\\
& & -	Server startet ohne Fehler\\
& & -Loglevel gel�scht (da nicht funktionst�chtig)\\
& & -Port kann mit Argument \grqq-p\grqq\ mitgegeben werden\\
& & -bei starten des Servers ohne Argumente kommt die Hilfeseite\\
& & -Das Kontroll-Strukt f�r das Shared Memory ist implementiert.\\
& & -Die Speicherverwaltung mit Buddy-System wurde beschlossen.\\
& & \ Das aufteilen der Bl�cke funktioniert einwandfrei\\
& & \ (wieder vereinen ist noch nicht implementiert)\\ \cline{1-3}

 \multirow{8}{1cm}{02.05.2014} &  \multirow{8}{1cm}{2.25h} & -Client TCP Connection zu Server aufbauen\\
& & -Client kann Verbindung aufbauen,\\
& & \ Message senden und Message erhalten.\\
& & \ Es fehlt jedoch ein Protokoll, dass die �bertragung sicherstellt.\\
& & -Teils werden noch zus�tzlich Zeichen angezeigt\\
& & \ (z.B. 25\$?d anstelle von 25)\\
& & -es gibt noch keine Validierung der Argumente\\
& & \ (z.B. CREATE, DELETE, \dots)\\  \cline{1-3}

 \multirow{5}{1cm}{03.05.2014} &  \multirow{5}{1cm}{6h} & -Log-Level implementiert mit verschiedenen Stufen.\\
 & & \ Output momentan nur m�glich auf CLI, jedoch mit Datum\\
 & & \ (z.B. May  3 2014 15:37:15: WARNING Test Log Warning)\\
& & - Implementierung von kleinem TCP Protokoll\\
& & \ (funktioniert nur beim Senden Client)\\     \cline{1-3}

 \multirow{4}{1cm}{04.05.2014} &  \multirow{4}{1cm}{6h} &-	�berpr�fung 1. Wort von Client als Command-Argument\\
 & & \ (Momentan nur Create File)\\ 
& & -Verfeinern CREATE Command\\
 & & - LOG verbessern\\ \cline{1-3}

 \multirow{7}{1cm}{05.05.2014} &  \multirow{7}{1cm}{8h} & -beim CREATE vom 2. File wurde der Name des ersten\\
 & & \ Files �berschrieben.\\
 & & \  Stundenlange Suche nach Ursache (Problem war ein zuweisen\\
 & & \ eines Pointer zum andren filename = filename.new anstelle\\
 & & \ filename = strdup(filename.new)\\
& & - Implementierung von LIST shm, was dem Client eine\\
& & \  komplette Liste des Shared Memory mit Adresse\\
& & \ und Dateinamen zur�ckliefert.\\ \cline{1-3}

 \multirow{7}{1cm}{06.05.2014} &  \multirow{7}{1cm}{5.5h} & -Dokumentation letzte 2 Tage\\
& & - Beheben von Warnings beim Kompilieren\\
& & -f�r das TRACE\_LOG k�nnen nur mehrere (dynamische\\
& & \  Variablen mitgeliefert werden.\\
& & -Problem, dass Server teils beim Erstellen eines Files abst�rzt.\\
& & \ Recherche im Internet: 1 Fehler war das malloc vor einen\\
& & \ strdup() -> Weniger Abst�rze, aber nicht ganz weg\\
& & - DELETE und READ fertig implementieren (ohne Lock)\\ \cline{1-3}

26.05.2014 & 1.75h & -Erstellen von Pthreads f�r Clients\\ \cline{1-3}

 \multirow{6}{1cm}{27.05.2014} &  \multirow{6}{1cm}{4.25h} & -Erstellen von PThreads f�r Client\\
 & & - Implementierung ReadWrite Lock mit pthread\_rwlock\_t\\
 & & \ (Momentan nur ReadLock beim Lesen)\\
 & & -Kommentieren von Code\\
 & & -l�schen von altem, nicht mehr benutztem Code\\
 & & -Anpassen Log-Design (damit besser lesbar)\\ \cline{1-3}

 \multirow{3}{1cm}{28.05.2014} &  \multirow{3}{1cm}{2h} & -Update Dokumentation\\
 & & -Anpassen Version Github/Dokumentation\\
 & & -Kapitel \ref{sec:Files und deren Funktionen} beginnen\\ \cline{1-3}


 \multirow{3}{1cm}{31.05.2014} &  \multirow{3}{1cm}{4.5h} & -Fehlerbehebung PThreadList\\
 & & - Senden von EXIT bei Beenden von Client an Server\\
 & & - Joining PThread nach Client-EXIT bei Server\\ \cline{1-3}

 \multirow{5}{1cm}{01.06.2014} &  \multirow{5}{1cm}{7h} &- Joining PThread nach Client-EXIT bei Server fertig\\
 & & -Probleme Segmentation Fault beim l�schen des letzten Files\\
 & & \  (mehrere Stunden Fehlersuche)\\
 & & \  -> Problem war Test.txt (fix in Code als Testfile)\\
& & -Implementierung von RW-Lock bei DELETE File\\ \cline{1-3}


\multirow{3}{1cm}{02.06.2014} &  \multirow{3}{1cm}{3.75h} & -Fehlerbehebung bei �bermittlung von gr�sseren Fileinhalten\\
& & -Code kommentieren\\
& & -Dokumentation erweitern\\ \cline{1-3}


\multirow{3}{1cm}{10.06.2014} &  \multirow{3}{1cm}{99h} & -Dokumentation anpassen und erweitern\\
& & -Server Parameter optimieren (-p, -l)\\
& & -Dokumentation Kapitel \ref{chapter:Anleitung}\\ \cline{1-3}



\caption{Aufw�nde Seminararbeit}
\label{tab:Aufw�nde}


\end{longtable} 









